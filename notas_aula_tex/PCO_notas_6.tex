\documentclass{article}
\usepackage{graphicx} % Required for inserting images
\usepackage{hyperref}
\usepackage[brazilian]{babel}
\usepackage{tikz, amsmath}
\usetikzlibrary{shapes.geometric, arrows}
\usepackage[paperheight=6.5in,paperwidth=4.8in,margin=0.1in]{geometry}
\tikzstyle{process} = [rectangle, minimum width=3em, minimum height=2em, text centered, draw=blue, fill=gray!10]
\tikzstyle{startend} = [ellipse, minimum width=2em, minimum height=1em, text centered, draw=red, fill=gray!10]
\usepackage[
    backend = biber,
    style = abnt,
    ]{biblatex}
\addbibresource{bibliografia.bib}
\usepackage[shortlabels]{enumitem}

\tikzstyle{arrow} = [thick,->,>=stealth]
\pagestyle{plain}

\title{%
 Processo constitucional \\
  \Large Notas de aula 6 \\
  \large Controle abstrato de constitucionalidade}
\author{Luccas Gissoni}
\date{6 de novembro 2024}

\begin{document}

\maketitle
\tableofcontents

\section{Introdução: modelos de controle de constitucionalidade}

O controle de constitucionalidade pode ser classificado de acordo com uma série de critérios \cite[cap. 2]{dimoulis_curso_2016}. Para nossos propósitos, os critérios mais importantes referem-se ao número de fiscais da constitucionalidade e 

\subsection{Quanto ao número de fiscais da constitucionalidade}

\subsubsection{Sistema difuso (universal)}

\begin{quote}
    Trata-se do também chamado modelo “americano”, por ser os Estados Unidos o primeiro país a adotá-lo. A fiscalização constitucional é realizada por todos os órgãos judiciais do ordenamento, sendo mais preciso denominá-lo universal.\\
    Cada órgão do Judiciário realiza o controle no âmbito de suas competências (...). Mas o importante é que não se proíbe que certo órgão judicial fiscalize a constitucionalidade. É o sistema adotado nos Estados Unidos, no Brasil, na Argentina, na Grécia e no Japão \cite[p. 78]{dimoulis_curso_2016}.
\end{quote}

\subsubsection{Sistema concentrado}

\begin{quote}
    Nesse caso, uma Corte Constitucional (ou Suprema Corte) decide sobre alegações de inconstitucionalidade, concentrando a competência. A Áustria, a Itália e a Alemanha são exemplos de controle concentrado.\\
    Mas nesses países a Corte Constitucional frequentemente decide após pedido de um tribunal que questiona a constitucionalidade de determinada lei. Se o Tribunal não apresentar o questionamento sobre a constitucionalidade, não haverá decisão da Corte. Dessa forma, os Tribunais acabam tendo uma importante, mas limitada, competência de controle de constitucionalidade. Não podem declarar a norma inconstitucional, mas podem suscitar a inconstitucionalidade provocando decisão da Corte Constitucional e, em alguns casos, também fazer interpretação conforme a Constituição para “salvar” a lei. Isso demonstra que a Corte Constitucional não é a única a decidir sobre o tema e faz a doutrina denominar esse sistema de “semiconcentrado” \cite[p. 79]{dimoulis_curso_2016}.
\end{quote}

\subsubsection{Sistemas mistos?}

\begin{quote}
    A distinção entre controle difuso e concentrado encontra-se descrita na obra de Carl Schmitt que, nos anos 1930, estudou a diferença entre o controle judicial estadunidense e o austríaco.26 Afirma-se que em nossos dias surgiu um terceiro modelo, que pode ser denominado híbrido ou misto, porque combina elementos dos anteriores. Teríamos tal sistema quando o Tribunal Supremo decide de forma concentrada sobre a constitucionalidade de leis, mas isso não impede que qualquer juiz possa realizar o controle de maneira difusa. Os autores se referem a sistema misto (difuso e concentrado),27 sendo que alguns preferem os termos dual/paralelo,28 combinado.29 São indicados como exemplos o Brasil, o México e Portugal (...).\\
    Mas isso não afeta a nossa classificação sobre o número dos fiscais da constitucionalidade. Quando se utiliza como critério de classificação o número de órgãos encarregados do controle, o sistema é difuso (universal) quando todos os órgãos do Judiciário realizam controle de constitucionalidade. Nesse sistema difuso, pode atuar uma Corte suprema com competências reforçadas e com capacidade para impor sua interpretação constitucional aos demais tribunais mediante decisões vinculantes. Mas isso diz respeito ao tipo de fiscalização (abstrato ou concreto) que será analisado em seguida. Do ponto de vista do número de fiscais, o sistema será difuso ou concentrado \cite[pp. 79-81]{dimoulis_curso_2016}.
\end{quote}

\subsection{Quanto ao tipo de fiscalização}

\subsubsection{Concreto}

\begin{quote}
    Nessa hipótese se decide sobre um caso concreto que demanda a aplicação de determinada norma infraconstitucional, verificando sua compatibilidade com preceitos constitucionais. Neste caso, o juiz soluciona apenas o litígio posto à sua apreciação. No controle concreto não há instauração de processo específico. A questão surge no decorrer de qualquer processo como forma de defesa processual dos interessados em determinada solução ou em razão da dúvida do julgador sobre a interpretação do direito.\\
    Quando, por exemplo, o devedor que deu em garantia seu único imóvel questiona a constitucionalidade da lei que possibilita a penhora, só pretende que a sua casa não seja penhorada. O pedido de declaração de inconstitucionalidade surge no decorrer do processo como um dos argumentos do interessado \cite[p. 84]{dimoulis_curso_2016}.
\end{quote}

\subsubsection{Abstrato}

\begin{quote}
    O controle promovido por ações diretas de (in)constitucionalidade tem como finalidade verificar a constitucionalidade de dispositivos em processo específico. É estruturado de acordo com as regras do denominado processo objetivo (Capítulo IV). Esse processo não se vincula juridicamente a interesses particulares a favor ou contra a inconstitucionalidade, nem a um caso concreto de litígio. A norma impugnada é analisada abstratamente, em sua relação de compatibilidade com a Constituição. A pergunta é se a norma está de acordo com a Constituição em tese, algo que pode se verificar ainda que a norma nunca tenha sido aplicada \cite[p. 84]{dimoulis_curso_2016}.
\end{quote}

\textit{Hoje estudaremos o controle abstrato de constitucionalidade}.

\section{Formas de inconstitucionalidade}

\begin{quote}
    Dizer que algo é inconstitucional é uma afirmação genérica. Para que a inconstitucionalidade possa ser processualmente verificada necessitamos diferenciar entre as suas formas, analisando se ocorre inconstitucionalidade de determinado tipo. Dois são os principais critérios para tanto. Primeiro, a natureza da norma constitucional violada. Segundo, o momento de ocorrência da inconstitucionalidade \cite[p. 96]{dimoulis_curso_2016}.
\end{quote}

\subsection{Natureza da norma violada}

\subsubsection{Inconstitucionalidade formal}

\begin{quote}
    Verifica-se quando na produção de certo dispositivo não são observados os requisitos de sua criação regular, em particular as normas constitucionais que definem regras de competência e o procedimento a ser observado para sua elaboração (no caso das leis federais: iniciativa, deliberação, votação, sanção ou veto, promulgação e publicação).\\
    Se, por exemplo, o Congresso propuser a edição de uma norma que for de iniciativa do Presidente da República, essa norma sofrerá do vício de inconstitucionalidade formal, ainda que seu conteúdo esteja em consonância com a Constituição e conte até com a concordância do próprio Presidente da República.\\
    Podemos distinguir entre inconstitucionalidade formal em razão de problemas no sujeito que decide (inconstitucionalidade por incompetência) ou em razão de defeitos no procedimento (inconstitucionalidade por desrespeito ao processo legislativo) \cite[p. 96]{dimoulis_curso_2016}.
\end{quote}

\subsubsection{Inconstitucionalidade material}

\begin{quote}
    Verifica-se quando o conteúdo de certo dispositivo contraria previsões de norma constitucional. Caso, por exemplo, uma lei estabeleça salário de servidores com valor superior àquele previsto no art. 37, XI, da Constituição Federal,\footnote{“A remuneração e o subsídio dos ocupantes de cargos, funções e empregos públicos da administração direta, (…) não poderão exceder o subsídio mensal, em espécie, dos Ministros do Supremo Tribunal Federal (…)”.} temos inconstitucionalidade de natureza material que se verifica mesmo se a norma tiver sido aprovada de maneira formalmente correta \cite[pp. 96-97]{dimoulis_curso_2016}.
\end{quote}

\subsection{Momento de ocorrência da inconstitucionalidade}

\subsubsection{Inconstitucionalidade originária}

\begin{quote}
    A inconstitucionalidade originária se verifica desde a entrada em vigor do dispositivo inconstitucional. É o caso mais simples e comum. Tendo, por exemplo, um dispositivo da Constituição de 1988 que dispõe que só brasileiros natos podem ser nomeados em certos cargos (art. 12, § 3o, da CF), eventual lei editada em 1989 que permite a nomeação de brasileiros naturalizados para esses cargos sofre de inconstitucionalidade desde a sua criação \cite[p. 97]{dimoulis_curso_2016}.
\end{quote}

\subsubsection{Inconstitucionalidade superveniente}

\begin{quote}
    A inconstitucionalidade pode afetar o dispositivo em momento posterior à sua criação. Isso significa que o dispositivo que estava em consonância com a Constituição no momento de sua criação passa a ter a pecha de inconstitucionalidade no decorrer de sua validade \cite[p. 97]{dimoulis_curso_2016}.
\end{quote}

\section{Ações de controle abstrato de constitucionalidade em espécie}

\begin{quote}
    Todas as ações de controle de constitucionalidade abstrato possuem um objeto muito claro e exposto na própria lei, não havendo qualquer complicação na identificação das peças.\\
    Partindo do princípio de que estas ações não se preocupam com casos específicos e concretos, e sim com o ordenamento jurídico, passamos a distinguir:\\
    A \textbf{ação direta de inconstitucionalidade} (ADI) tem por fim declarar a inconstitucionalidade de norma federal, estadual ou até mesmo distrital quando no exercício de competência legislativa estadual. Não é necessário que haja controvérsia judicial, porém, obrigatório que a lei esteja em vigor e tenha sido publicada após a promulgação da Constituição da República.\\
    A \textbf{ação declaratória de constitucionalidade} (ADC), por sua vez, busca confirmar a constitucionalidade de norma federal (somente federal) que esteja sendo questionada por diversas decisões judiciais contraditórias. Nesse caso, tem que existir controvérsia judicial e a norma posterior à Constituição há de ter natureza federal, não cabendo esta ação para normas estaduais, distritais ou municipais.\\
    A \textbf{ação direta de inconstitucionalidade por omissão} (ADIO) tem por objetivo cientificar o poder competente da inércia legislativa que impossibilita o exercício de direitos constitucionais previstos em norma constitucional de eficácia limitada. O responsável pela inércia poderá ser tanto o Poder Executivo como o Poder Legislativo federal e estadual ou até mesmo distrital, desde que diga respeito à competência legislativa estadual.\\
    Por fim, a \textbf{arguição de descumprimento de preceito fundamental} (ADPF), que possui um objeto extremamente abrangente, pois sua função é servir para todas as hipóteses em que não será cabível outra ação, apresentando como uma de suas características a subsidiariedade.\\
    Esta última ação será proposta com a intenção de evitar ou reparar lesão a preceito fundamental (p. ex., arts. 1.º ao 17, 34, VII, 37, caput, 60, § 4.º, da CR) resultante de ato do Poder Público, bem como resolver controvérsia constitucional a respeito de norma federal, estadual, distrital ou municipal, seja anterior ou posterior à Constituição.\\
    Pode parecer, pelo conceito extremamente abrangente, que a ADPF tenha objeto similar a outras ações acima transcritas, mas na verdade, por sua subsidiariedade, como afirmado, ela só caberá quando não houver outro meio eficaz de sanar a lesividade \cite[n. p.]{padilha_oab_2015}
\end{quote}

\subsection{Ação direta de inconstitucionalidade (ADI ou ADIn)}

\begin{quote}
    Objetivo geral da ADIn é impedir que norma contrária à Constituição permaneça no ordenamento jurídico, comprometendo a regularidade do sistema normativo por violar a supremacia constitucional. Procura-se, dessa forma, assegurar que a norma constitucional será imposta inclusive em relação aos poderes estatais. A aplicação de atos normativos inconstitucionais que costumam ter grande repercussão social gera danos de difícil reparação, sendo recomendada sua eliminação célere e definitiva.\\
    Além de preservar a supremacia constitucional, a ADIn, tal como as demais ações do controle de constitucionalidade abstrato, objetiva preservar a segurança jurídica, impedindo que surjam decisões discrepantes sobre a constitucionalidade.\\
    Processualmente a ADIn é um meio para realizar o controle de constitucionalidade de tipo judicial, abstrato e repressivo. Quando há decisão de mérito, no fim do processo, se declara a inconstitucionalidade de certo dispositivo, determinando sua nulidade ou, quando improcedente a ação, confirma-se a obrigatoriedade do ato questionado. Isso permite eliminar incertezas geradas por controvérsias jurídicas acerca da constitucionalidade de normas.\\
    Esta forma de controle de constitucionalidade foi introduzida no ordenamento brasileiro pela Emenda Constitucional no 16, de 26-11-1965, que instituiu a Representação contra inconstitucionalidade, a ser encaminhada ao STF pelo Procurador-Geral da República. A Constituição de 1988 instituiu a ADIn e atribuiu a competência para o seu julgamento ao STF (art. 102, I, a, da CF).\\
    Na ausência de previsões constitucionais e legais detalhadas, o procedimento da ADIn foi delineado pelo STF mediante autocriação de normas processuais (Capítulo IV, 5). A regulamentação legal do rito da ADIn veio com a Lei 9.868, de 1999, que incorporou muitos entendimentos do STF \cite[pp. 104-105]{dimoulis_curso_2016}.
\end{quote}

\subsubsection{Normas vigentes}

\begin{quote}
    Art. 102. Compete ao Supremo Tribunal Federal, precipuamente, a guarda da Constituição, cabendo-lhe:\\
    I – processar e julgar, originariamente:\\
    a) a ação direta de inconstitucionalidade de lei ou ato normativo federal ou estadual (...).
    p) o pedido de medida cautelar das ações diretas de inconstitucionalidade;\\
    Art. 103. Podem propor a ação direta de inconstitucionalidade […]:\\
    I – o Presidente da República;\\
    II – a Mesa do Senado Federal;\\
    III – a Mesa da Câmara dos Deputados;\\
    IV – a Mesa de Assembleia Legislativa ou da Câmara Legislativa do Distrito Federal;\\
    V – o Governador de Estado ou do Distrito Federal;\\
    VI – o Procurador-Geral da República;\\
    VII – o Conselho Federal da Ordem dos Advogados do Brasil;\\
    VIII – partido político com representação no Congresso Nacional;\\
    IX – confederação sindical ou entidade de classe de âmbito nacional.\\
    § 1o O Procurador-Geral da República deverá ser previamente ouvido nas ações de inconstitucionalidade e em todos os processos de competência do Supremo Tribunal Federal (...).\\
    § 3o Quando o Supremo Tribunal Federal apreciar a inconstitucionalidade, em tese, de norma legal ou ato normativo, citará, previamente, o Advogado-Geral da União, que defenderá o ato ou texto impugnado \cite{brasil_constituicao_1988}.
\end{quote}

\begin{quote}
    Lei 9.868, de 10-11-1999

Dispõe sobre o processo e julgamento da ação direta de inconstitucionalidade e da ação declaratória de constitucionalidade perante o Supremo Tribunal Federal.

O PRESIDENTE DA REPÚBLICA

Faço saber que o Congresso Nacional decreta e eu sanciono a seguinte Lei: CAPÍTULO I

Da Ação Direta de Inconstitucionalidade e da Ação Declaratória de Constitucionalidade

Art. 1o Esta Lei dispõe sobre o processo e julgamento da ação direta de inconstitucionalidade e da ação declaratória de constitucionalidade perante o Supremo Tribunal Federal.

CAPÍTULO II

Da Ação Direta de Inconstitucionalidade

Seção I

Da Admissibilidade e do Procedimento da Ação Direta de Inconstitucionalidade

Art. 2o Podem propor a ação direta de inconstitucionalidade:

I – o Presidente da República;

II – a Mesa do Senado Federal;

III – a Mesa da Câmara dos Deputados;

IV – a Mesa de Assembleia Legislativa ou a Mesa da Câmara Legislativa do Distrito Federal;

V – o Governador de Estado ou o Governador do Distrito Federal;

VI – o Procurador-Geral da República;

VII – o Conselho Federal da Ordem dos Advogados do Brasil;

VIII – partido político com representação no Congresso Nacional;

IX – confederação sindical ou entidade de classe de âmbito nacional. Art. 3o A petição indicará:

I – o dispositivo da lei ou do ato normativo impugnado e os fundamentos jurídicos do pedido em relação a cada uma das impugnações;

II – o pedido, com suas especificações.

Parágrafo único. A petição inicial, acompanhada de instrumento de procuração, quando subscrita por advogado, será apresentada em duas vias, devendo conter cópias da lei ou do ato normativo impugnado e dos documentos necessários para comprovar a impugnação.

Art. 4o A petição inicial inepta, não fundamentada e a manifestamente improcedente serão liminarmente indeferidas pelo relator.

Parágrafo único. Cabe agravo da decisão que indeferir a petição inicial. Art. 5o Proposta a ação direta, não se admitirá desistência.

Art. 6o O relator pedirá informações aos órgãos ou às autoridades das quais emanou a lei ou o ato normativo impugnado.

Parágrafo único. As informações serão prestadas no prazo de trinta dias contado do recebimento do pedido.

Art. 7o Não se admitirá intervenção de terceiros no processo de ação direta de inconstitucionalidade.

§ 2o O relator, considerando a relevância da matéria e a representatividade dos postulantes, poderá, por despacho irrecorrível, admitir, observado o prazo fixado no parágrafo anterior, a manifestação de outros órgãos ou entidades.

Art. 8o Decorrido o prazo das informações, serão ouvidos, sucessivamente, o Advogado-Geral da União e o Procurador-Geral da República, que deverão manifestar-se, cada qual, no prazo de quinze dias.

Art. 9o Vencidos os prazos do artigo anterior, o relator lançará o relatório, com cópia a todos os Ministros, e pedirá dia para julgamento.

§ 1o Em caso de necessidade de esclarecimento de matéria ou circunstância de fato ou de notória insuficiência das informações existentes nos autos, poderá o relator requisitar informações adicionais, designar perito ou comissão de peritos para que emita parecer sobre a questão, ou fixar data para, em audiência pública, ouvir depoimentos de pessoas com experiência e autoridade na matéria.

§ 2o O relator poderá, ainda, solicitar informações aos Tribunais Superiores, aos Tribunais federais e aos Tribunais estaduais acerca da aplicação da norma impugnada no âmbito de sua jurisdição.

§ 3o As informações, perícias e audiências a que se referem os parágrafos anteriores serão realizadas no prazo de trinta dias, contado da solicitação do relator.

Seção II

Da Medida Cautelar em Ação Direta de Inconstitucionalidade

Art. 10. Salvo no período de recesso, a medida cautelar na ação direta será concedida por decisão da maioria absoluta dos membros do Tribunal, observado o disposto no art. 22, após a audiência dos órgãos ou autoridades dos quais emanou a lei ou ato normativo impugnado, que deverão pronunciar-se no prazo de cinco dias.

§ 1o O relator, julgando indispensável, ouvirá o Advogado-Geral da União e o Procurador-Geral da República, no prazo de três dias.

§ 2o No julgamento do pedido de medida cautelar, será facultada sustentação oral aos representantes judiciais do requerente e das autoridades ou órgãos responsáveis pela expedição do ato, na forma estabelecida no Regimento do Tribunal.

§ 3o Em caso de excepcional urgência, o Tribunal poderá deferir a medida cautelar sem a audiência dos órgãos ou das autoridades das quais emanou a lei ou o ato normativo impugnado.

Art. 11. Concedida a medida cautelar, o Supremo Tribunal Federal fará publicar em seção especial do Diário Oficial da União e do Diário da Justiça da União a parte dispositiva da decisão, no prazo de dez dias, devendo solicitar as informações à autoridade da qual tiver emanado o ato, observando-se, no que couber, o procedimento estabelecido na Seção I deste Capítulo.

§ 1o A medida cautelar, dotada de eficácia contra todos, será concedida com efeito ex nunc, salvo se o Tribunal entender que deva conceder-lhe eficácia retroativa.

§ 2o A concessão da medida cautelar torna aplicável a legislação anterior acaso existente, salvo expressa manifestação em sentido contrário.

Art. 12. Havendo pedido de medida cautelar, o relator, em face da relevância da matéria e de seu especial significado para a ordem social e a segurança jurídica, poderá, após a prestação das informações, no prazo de dez dias, e a manifestação do Advogado-Geral da União e do Procurador-Geral da República, sucessivamente, no prazo de cinco dias, submeter o processo diretamente ao Tribunal, que terá a faculdade de julgar definitivamente a ação.

CAPÍTULO IV

Da Decisão na Ação Direta de Inconstitucionalidade e

na Ação Declaratória de Constitucionalidade

Art. 22. A decisão sobre a constitucionalidade ou a inconstitucionalidade da lei ou do ato normativo somente será tomada se presentes na sessão pelo menos oito Ministros.

Art. 23. Efetuado o julgamento, proclamar-se-á a constitucionalidade ou a inconstitucionalidade da disposição ou da norma impugnada se num ou noutro sentido se tiverem manifestado pelo menos seis Ministros, quer se trate de ação direta de inconstitucionalidade ou de ação declaratória de constitucionalidade.

Parágrafo único. Se não for alcançada a maioria necessária à declaração de constitucionalidade ou de inconstitucionalidade, estando ausentes Ministros em número que possa influir no julgamento, este será suspenso a fim de aguardar-se o comparecimento dos Ministros ausentes, até que se atinja o número necessário para prolação da decisão num ou noutro sentido.

Art. 24. Proclamada a constitucionalidade, julgar-se-á improcedente a ação direta ou procedente eventual ação declaratória; e, proclamada a inconstitucionalidade, julgar-se-á procedente a ação direta ou improcedente eventual ação declaratória.

Art. 25. Julgada a ação, far-se-á a comunicação à autoridade ou ao órgão responsável pela expedição do ato \cite{brasil_lei9868_1999}.
\end{quote}

\subsubsection{Modelo de petição inicial}

\begin{quote}
    EXCELENTÍSSIMO SENHOR DOUTOR MINISTRO PRESIDENTE DO SUPREMO TRIBUNAL FEDERAL\\


(Pessoa relacionada no art. 103 da CRFB/1988 e art. 2.º da Lei 9.868/1999) vem, por meio de seu advogado que esta subscreve, com instrumento de mandato anexo e endereço constante no rodapé da presente, para onde devem ser remetidas as intimações, nos moldes do art. 39, I, do CPC, respeitosamente, perante Vossa Excelência, nos termos do art. 103, inciso (x), art. 102, I, a e p, ambos da CRFB/1988, arts. 2.º, inciso (x) e 10 da Lei 9.868/1999 e arts. 282 e ss. do CPC, propor a presente\\

\textit{\textbf{NOTA}: Quase todos possuem capacidade postulatória plena, não precisando de advogado (só se vale de advogado se quiser). Os únicos do art. 103 da CRFB de quem se exige representação por advogado são os incs. VIII e IX da CRFB, ou seja, partido político com representação no Congresso Nacional, confederação sindical e entidade de classe de âmbito nacional.}\\

\begin{center}
    AÇÃO DIRETA DE INCONSTITUCIONALIDADE
\end{center}

tendo por objeto a declaração de inconstitucionalidade da Lei (x), pelos fatos e fundamentos jurídicos que passa a expor:\\

\textit{\textbf{NOTA}: Apesar de a ADI ser uma ação objetiva, onde não há partes, o STF também admite ADI que indique o legitimado passivo. Nestes moldes, deve constar: “em face de (x) (órgão ou autoridade que editou o ato que se pretende impugnar e endereço – se possível), pelos fatos e fundamentos jurídicos que passa a expor:”}\\

I – DA PERTINÊNCIA TEMÁTICA\\

\textit{\textbf{NOTA}: Neste tópico deve-se demonstrar a legitimidade do impetrante (que deve ser um dos elencados nos art. 103 da CRFB/1988). Lembre-se de que de alguns entes se exige \textbf{pertinência} temática (chamados autores interessados ou especiais). \textbf{Então esse é o momento de demonstrar a relação que possui com a norma que se pretende impugnar}.\\
A prova da pertinência é exigida quando a ADI é proposta pelo Governador do Estado ou do Distrito Federal, Mesa da Assembleia Legislativa ou da Câmara Legislativa do Distrito Federal, confederações sindicais ou entidades de âmbito nacional.\\
Os demais legitimados são chamados de autores neutros ou universais, pois não se exige a demonstração de pertinência.}\\

II – DOS FATOS\\

\textit{\textbf{NOTA}: A comprovação da matéria de fato só deve ser arguida quando o problema demonstrar inconstitucionalidade formal, quando houver vício na iniciativa ou no procedimento, tendo de ser relatado o fato.\\
Porém, se a inconstitucionalidade for material não se justifica discorrer sobre os fatos, uma vez que a questão é unicamente de direito.\\
\textbf{ATENÇÃO}: Quando for necessário relatar fatos, devem ser descritas somente as questões trazidas na questão da prova, sendo proibido inventar situações não descritas, sob pena de identificação de peça.}\\

III – DA INCONSTITUCIONALIDADE\\

\textit{\textbf{NOTA}: Ao fundamentar a inconstitucionalidade, deve-se atentar a:}

\begin{itemize}
    \item \textit{Constituição da República;}
    \item \textit{Princípios;}
    \item \textit{Súmula;}
    \item \textit{Doutrina;}
    \item \textit{Jurisprudência.}\\
\end{itemize}
\\

IV – DA MEDIDA LIMINAR\\

\textit{\textbf{NOTA}: Esta medida está prevista no art. 10 da Lei 9.868/1999 e deverá abordar:}

\begin{itemize}
    \item \textit{O perigo na demora da prestação jurisdicional (periculum in mora), em razão do possível dano irreparável causado caso a liminar não seja deferida.}
    \item \textit{Verossimilhança da alegação (fumus boni iuris).}
\end{itemize}\\

V – DO PEDIDO\\

Ante o exposto, requer:\\

a) a concessão de medida liminar, para suspender a eficácia do dispositivo questionado na forma do art. 10 da Lei 9.868/1999;

b) a notificação do ente responsável pela elaboração da norma para, querendo, se manifestar no prazo legal (ou no prazo de 30 dias);

c) a notificação do Advogado-Geral da União, para se manifestar sobre o mérito da presente ação, no prazo legal (ou no prazo de 15 dias);

d)a notificação do Procurador-Geral da República, para emitir seu parecer no prazo legal (ou no prazo de 15 dias);

e)a procedência do pedido de mérito para que seja declarada a inconstitucionalidade do art. (x) da Lei (x) (citar o dispositivo).\\

\textit{\textbf{NOTA}: Quando a alegação for sobre inconstitucionalidade formal, deve existir o requerimento de provas da seguinte forma:}\\

\textit{“Pretende produzir todos os meios de prova em direito admitidos, em especial documental e oral.”}\\

Atribui a causa o valor de RS (valor por extenso).\\

Nesses termos,\\

Espera deferimento.\\

 

Local e data.

Advogado/OAB\\
\cite[n. p.]{padilha_oab_2015}
\end{quote}

\subsubsection{Exercício de fixação}

\begin{quote}
    Partido Honestidade, Partido Político que possui um Deputado Federal na Câmara dos Deputados, procura você, advogado, buscando solução quanto a Lei Nacional 8.000/1989, que, por iniciativa do Deputado Federal Astolfo Rocha, membro do partido político Esquerda Nunca Mais, organizou internamente o Ministério Público dos Estados. Na qualidade de advogado do Partido da Honestidade, redija a peça processual cabível \cite{padilha_oab_2015}.
\end{quote}

\printbibliography

\end{document}