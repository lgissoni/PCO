\documentclass{article}
\usepackage{graphicx} % Required for inserting images
\usepackage{hyperref}
\usepackage[brazilian]{babel}
\usepackage{tikz, amsmath}
\usetikzlibrary{shapes.geometric, arrows}
\usepackage[paperheight=6.5in,paperwidth=4.8in,margin=0.1in]{geometry}
\tikzstyle{process} = [rectangle, minimum width=3em, minimum height=2em, text centered, draw=blue, fill=gray!10]
\tikzstyle{startend} = [ellipse, minimum width=2em, minimum height=1em, text centered, draw=red, fill=gray!10]
\usepackage[
    backend = biber,
    style = abnt,
    ]{biblatex}
\addbibresource{bibliografia.bib}
\usepackage[shortlabels]{enumitem}

\tikzstyle{arrow} = [thick,->,>=stealth]
\pagestyle{plain}

\title{%
 Processo constitucional \\
  \Large Notas de aula 4 \\
  \large Processo legislativo}
\author{Luccas Gissoni}
\date{18 de setembro 2024}

\begin{document}

\maketitle
\tableofcontents

\section{Introdução}

É função típica do Poder Legislativo editar \textbf{atos normativos primários}, isto é, aqueles que instituem direitos e criam obrigações.

\begin{quote}
    Art. 59. O processo legislativo compreende a elaboração de:\\
    I - emendas à Constituição;\\
    II - leis complementares;\\
    III - leis ordinárias;\\
    IV - leis delegadas;\\
    V - medidas provisórias;\\
    VI - decretos legislativos;\\
    VII - resoluções.\\
    Parágrafo único. Lei complementar disporá sobre a elaboração, redação, alteração e consolidação das leis \cite{brasil_constituicao_1988}. 
\end{quote}

\section{A iniciativa}

\begin{quote}
    O processo legislativo tem início quando alguém ou algum ente toma a iniciativa de apresentar uma proposta de criação de novo direito \cite[p.~156]{mendes_curso_2024}.
\end{quote}

Em regra, o processo legislativo tem início na Câmara dos Deputados, a menos que resulte de iniciativa de senador ou de comissão do Senado Federal.

\subsection{Iniciativa comum}

\begin{quote}
    Art. 61. A iniciativa das leis complementares e ordinárias cabe a qualquer membro ou Comissão da Câmara dos Deputados, do Senado Federal ou do Congresso Nacional, ao Presidente da República, ao Supremo Tribunal Federal, aos Tribunais Superiores, ao Procurador-Geral da República e aos cidadãos, na forma e nos casos previstos nesta Constituição \cite{brasil_constituicao_1988}.
\end{quote}

\begin{quote}
    A iniciativa é dita comum (ou concorrente) se a proposição normativa puder ser apresentada por qualquer membro do Congresso Nacional ou por comissão de qualquer de suas Casas, bem assim pelo Presidente da República, e, ainda, pelos cidadãos, no caso da iniciativa popular (CF, art. 61, § 2º). A iniciativa é comum para as proposições em que o constituinte não tenha restringido o âmbito da sua titularidade \cite[p.~156]{mendes_curso_2024}.
\end{quote}

\subsection{Iniciativa reservada}

\begin{quote}
    Em algumas hipóteses, a Constituição reserva a possibilidade de dar início ao processo legislativo a apenas algumas autoridades ou órgãos. Fala-se, então, em iniciativa reservada ou privativa. Como figuram hipóteses de exceção, os casos de iniciativa reservada não devem ser ampliados por via interpretativa.\\
    A iniciativa privativa visa subordinar ao seu titular a conveniência e oportunidade da deflagração do debate legislativo em torno do assunto reservado \cite[p.~156]{mendes_curso_2024}.
\end{quote}

\subsubsection{Iniciativa privativa de órgãos do Judiciário}

É de competência privativa:

\paragraph{Do Supremo Tribunal Federal} Propor lei complementar sobre o Estatuto da Magistratura.

\begin{quote}
    Art. 93. Lei complementar, de iniciativa do Supremo Tribunal Federal, disporá sobre o Estatuto da Magistratura, observados os seguintes princípios... \cite{brasil_constituicao_1988}
\end{quote}

\paragraph{Dos tribunais} Propor a criação de novas varas judiciárias em sua própria estrutura.

\begin{quote}
    Art. 96. Compete privativamente:\\
    I - aos tribunais:\\
    (...)\\
    d) propor a criação de novas varas judiciárias... \cite{brasil_constituicao_1988}
\end{quote}

\paragraph{Do Supremo Tribunal Federal, dos Tribunais Superiores e dos Tribunais de Justiça} Propor a estruturação e organização dos tribunais inferiores.

\begin{quote}
    Art. 96. Compete privativamente:\\
    (...)\\
    II - ao Supremo Tribunal Federal, aos Tribunais Superiores e aos Tribunais de Justiça propor ao Poder Legislativo respectivo, observado o disposto no art. 169:\\
    a) a alteração do número de membros dos tribunais inferiores;\\
    b) a criação e a extinção de cargos e a remuneração dos seus serviços auxiliares e dos juízos que lhes forem vinculados, bem como a fixação do subsídio de seus membros e dos juízes, inclusive dos tribunais inferiores, onde houver;\\
    c) a criação ou extinção dos tribunais inferiores;\\
    d) a alteração da organização e da divisão judiciárias...  \cite{brasil_constituicao_1988}
\end{quote}

\subsubsection{Iniciativa privativa do Ministério Público}

Compete ao Ministério Público a iniciativa para propor ao Poder Legislativo a criação e extinção de seus cargos e serviços auxiliares, bem assim a política remuneratória e os planos de carreira.

\begin{quote}
    Art. 127. O Ministério Público é instituição permanente, essencial à função jurisdicional do Estado, incumbindo-lhe a defesa da ordem jurídica, do regime democrático e dos interesses sociais e individuais indisponíveis.\\
    (...)\\
    § 2º Ao Ministério Público é assegurada autonomia funcional e administrativa, podendo, observado o disposto no art. 169, propor ao Poder Legislativo a criação e extinção de seus cargos e serviços auxiliares, provendo-os por concurso público de provas ou de provas e títulos, a política remuneratória e os planos de carreira; a lei disporá sobre sua organização e funcionamento... \cite{brasil_constituicao_1988}
\end{quote}

\subsubsection{Iniciativa privativa da Câmara dos Deputados, do Senado e do Tribunal de Contas da União}

A Câmara dos Deputados e o Senado Federal têm a iniciativa privativa para leis que fixem a remuneração dos servidores incluídos na sua organização.

\paragraph{Câmara dos Deputados}

\begin{quote}
    Art. 51. Compete privativamente à Câmara dos Deputados:\\
    (...)\\
    IV - dispor sobre sua organização, funcionamento, polícia, criação, transformação ou extinção dos cargos, empregos e funções de seus serviços, e a iniciativa de lei para fixação da respectiva remuneração, observados os parâmetros estabelecidos na lei de diretrizes orçamentárias... \cite{brasil_constituicao_1988}
\end{quote}

\paragraph{Senado Federal}

\begin{quote}
    Art. 52. Compete privativamente ao Senado Federal:\\
    (...)\\
    XIII - dispor sobre sua organização, funcionamento, polícia, criação, transformação ou extinção dos cargos, empregos e funções de seus serviços, e a iniciativa de lei para fixação da respectiva remuneração, observados os parâmetros estabelecidos na lei de diretrizes orçamentárias... \cite{brasil_constituicao_1988}
\end{quote}

\paragraph{Tribunal de Contas da União}

\begin{quote}
    Para preservar a independência orgânica do Tribunal de Contas da União, o constituinte estendeu-lhe o exercício das atribuições previstas no art. 96 do Texto Constitucional, que cuida também da iniciativa reservada de lei por parte de órgãos do Judiciário. Assim, o TCU tem iniciativa para apresentar projeto de lei visando a dispor sobre a sua organização administrativa, criação de cargos e remuneração de servidores, e fixação de subsídios dos membros da Corte \cite[p.~1057]{mendes_curso_2024}.
\end{quote}

\subsubsection{Iniciativa privativa do Presidente da República}

A Constituição reservou um extenso rol de matérias à iniciativa privativa do Presidente da República. A lógica é a seguinte:

\paragraph{Temas relacionados ao regime jurídico de servidores públicos, civis e militares}

\begin{quote}
    Art. 61. (...)\\
    § 1º São de iniciativa privativa do Presidente da República as leis que:\\
    I - fixem ou modifiquem os efetivos das Forças Armadas;\\
    II - disponham sobre:\\
    a) criação de cargos, funções ou empregos públicos na administração direta e autárquica ou aumento de sua remuneração;\\
    b) organização administrativa e judiciária, matéria tributária e orçamentária, serviços públicos e pessoal da administração dos Territórios;\\
    c) servidores públicos da União e Territórios, seu regime jurídico, provimento de cargos, estabilidade e aposentadoria;\\
    d) organização do Ministério Público e da Defensoria Pública da União, bem como normas gerais para a organização do Ministério Público e da Defensoria Pública dos Estados, do Distrito Federal e dos Territórios;\\
    e) criação e extinção de Ministérios e órgãos da administração pública, observado o disposto no art. 84, VI;\\
    f) militares das Forças Armadas, seu regime jurídico, provimento de cargos, promoções, estabilidade, remuneração, reforma e transferência para a reserva... \cite{brasil_constituicao_1988}
\end{quote}

Assim, é de iniciativa privativa do Chefe do Executivo leis que versem sobre as atribuições dos órgãos e cargos da administração e os requisitos para seu preenchimento, bem como aumento de vencimentos ou criação de vantagens. ``Disposições normativas sobre organização e funcionamento da Administração Federal, que não impliquem aumento de despesa, passaram a ser objeto de decreto do Presidente da República" \cite[p.~1057]{mendes_curso_2024}

\paragraph{Leis orçamentárias}

\begin{quote}
    Art. 84. Compete privativamente ao Presidente da República:\\
    (...)\\
    XXIII - enviar ao Congresso Nacional o plano plurianual, o projeto de lei de diretrizes orçamentárias e as propostas de orçamento previstos nesta Constituição... \cite{brasil_constituicao_1988}
\end{quote}

\begin{quote}
    Art. 165. Leis de iniciativa do Poder Executivo estabelecerão:\\
    I - o plano plurianual;\\
    II - as diretrizes orçamentárias;\\
    III - os orçamentos anuais \cite{brasil_constituicao_1988}.
\end{quote}

Neste caso, temos uma iniciativa \textbf{reservada e vinculada}, pois a apresentação da proposta é obrigatória.

\begin{quote}
    Matéria tributária não se insere no âmbito da iniciativa reservada do Presidente da República. O art. 61, § 1º, II, b, fala em matéria tributária, mas aquela relacionada aos Territórios apenas. A lei que concede benefício tributário, assim, não é da iniciativa reservada do Chefe do Executivo, não cabendo cogitar, aqui, da repercussão no orçamento dela decorrente, já que ``a iniciativa reservada, por constituir matéria de direito estrito, não se presume e nem comporta interpretação ampliativa” \cite[p.~1058]{mendes_curso_2024}.
\end{quote}

\section{Discussão}

Na fase de deliberação parlamentar, ``projeto de lei seguirá, na respectiva Casa Legislativa, para a fase da instrução, nas comissões (CF, art. 58, § 2º, I), onde será analisada inicialmente sua constitucionalidade e posteriormente seu mérito, nas chamadas, respectivamente, Comissão de Constituição e Justiça e Comissões Temáticas" \cite[p.~766]{moraes_direito_2023}.

\begin{quote}
    Depois de apresentado, o projeto é debatido nas comissões e nos plenários das Casas Legislativas. Podem ser formuladas emendas (proposições alternativas) aos projetos. A emenda cabe ao parlamentar e, em alguns casos, sofre restrições.\\
    Não se admite a proposta de emenda que importe aumento de despesa prevista nos projetos de iniciativa exclusiva do Presidente da República e nos projetos sobre organização dos serviços administrativos da Câmara dos Deputados, do Senado Federal, dos Tribunais Federais e do Ministério Público (CF, art. 63 e incisos). Assim, não se impede a emenda em casos de iniciativa reservada, mas a emenda estará vedada se importar incremento de dispêndio.\\
    Nos casos de leis que cuidam de matéria orçamentária, é também possível a emenda parlamentar, mas com certas ressalvas. Nas leis de orçamento anual, as emendas devem ser compatíveis com o plano plurianual e com a lei de diretrizes orçamentárias. Devem, ainda, indicar os recursos necessários para atendê-las, por meio de anulação de outras despesas previstas no projeto. Não podem ser anuladas despesas previstas para dotações para pessoal e seus encargos, serviço da dívida e transferências tributárias constitucionais para Estados, Municípios e Distrito Federal, nos termos do art. 166, § 3º, da Constituição. O § 4º do mesmo dispositivo cobra a compatibilidade da emenda ao projeto de diretrizes orçamentárias com o plano plurianual.\\
    O STF entende que, a par dessa limitação expressa ao direito de emendar projeto da iniciativa reservada do Chefe do Executivo, outra mais deve ser observada, por consequência lógica do sistema – a emenda deve guardar pertinência com o projeto de iniciativa privativa, para prevenir a fraude a essa mesma reserva. A pertinência da emenda com o projeto de iniciativa reservada deve ser estreita. Deve-se levar em conta os limites materiais do temário do projeto e o seu propósito, a fim de se apurar a viabilidade da emenda parlamentar, mesmo que não importe aumento de despesa \cite[p.~1060]{mendes_curso_2024}
\end{quote}

\section{Votação}

Após o período de debates, o projeto segue para votação. Para aprovação, deverá ser atingido o quorum mínimo exigido para a matéria pela Constituição ou, em seu silêncio, a maioria simples.

Uma vez aprovado pela Casas Legislativa em que teve início sua tramitação, o projeto segue para a outra Casa, que atuará como revisora (art. 65), e passará novamente pelo procedimento de análise nas comissões e votação. Daí, se aprovado, segue para sanção ou veto do Presidente da República (próxima seção). Se, contudo, for aprovado com alterações, retornará à Casa inicial, para que estas sejam apreciadas segundo o mesmo procedimeto: discussão nas comissões e votação. Se o projeto for reijeitado pela Casa Revisora, será arquivado.

\begin{quote}
    Art. 65. O projeto de lei aprovado por uma Casa será revisto pela outra, em um só turno de discussão e votação, e enviado à sanção ou promulgação, se a Casa revisora o aprovar, ou arquivado, se o rejeitar.\\
    Parágrafo único. Sendo o projeto emendado, voltará à Casa iniciadora.\\
    Art. 66. A Casa na qual tenha sido concluída a votação enviará o projeto de lei ao Presidente da República, que, aquiescendo, o sancionará... \cite{brasil_constituicao_1988}
\end{quote}

\begin{quote}
    Importante ressaltar que em face do princípio do bicameralismo, qualquer emenda ao projeto aprovado por uma das Casas, haverá, obrigatoriamente, que retornar à outra, para que se pronuncie somente sobre esse ponto, para aprová-lo ou rejeitá-lo, de forma definitiva. Dessa forma, o posicionamento da Casa que iniciar o processo legislativo (Deliberação Principal) prevalecerá nesta hipótese \cite[p.~768]{moraes_direito_2023}.
\end{quote}

\paragraph{Regime de urgência}

\begin{quote}
    Art. 64. A discussão e votação dos projetos de lei de iniciativa do Presidente da República, do Supremo Tribunal Federal e dos Tribunais Superiores terão início na Câmara dos Deputados.\\
    § 1º - O Presidente da República poderá solicitar urgência para apreciação de projetos de sua iniciativa.\\\\
    § 2º Se, no caso do § 1º, a Câmara dos Deputados e o Senado Federal não se manifestarem sobre a proposição, cada qual sucessivamente, em até quarenta e cinco dias, sobrestar-se-ão todas as demais deliberações legislativas da respectiva Casa, com exceção das que tenham prazo constitucional determinado, até que se ultime a votação.\\
    § 3º A apreciação das emendas do Senado Federal pela Câmara dos Deputados far-se-á no prazo de dez dias, observado quanto ao mais o disposto no parágrafo anterior.\\
    § 4º Os prazos do § 2º não correm nos períodos de recesso do Congresso Nacional, nem se aplicam aos projetos de código. 
\end{quote}

\section{Sanção e veto}

O Presidente da República não participa do processo legislativo apenas quando propõe projetos d sua iniciativa, mas também quando, terminada a fase de votação, é chamado a sancionar ou vetar o projeto aprovado pelo Legislativo.

\subsection{Sanção}

Concordando com o projeto, o Presidente pode sancioná-lo \textbf{expressamente}, ou \textbf{tacitamente}, quando não o veta no prazo constitucional.

\begin{quote}
    O STF entendeu, no passado, que a sanção ao projeto que surgiu de usurpação da iniciativa privativa do Presidente da República sanava o vício, suprindo a falta da iniciativa correta (Súmula 5/STF). A Súmula 5 foi objeto de críticas diversas, como a de que ela não atentaria para que o vício de inconstitucionalidade ocorrido em uma etapa do processo legislativo contamina de nulidade inconvalidável a lei que dele surge, bem assim a de que o Presidente da República não pode desvestir-se das prerrogativas que a Constituição lhe assina. Objetou-se, mais, que a tese sumulada pode provocar o embaraço político ao Chefe do Executivo, o que a Constituição quis precisamente evitar, ao lhe reservar a iniciativa do projeto. A Súmula, afinal, embora nunca tenha sido formalmente cancelada, foi sendo relegada na prática, até que se firmou que a inteligência sumulada não é mais aplicável. Portanto, hoje, tem-se por certo que mesmo vindo o Chefe do Executivo a sancionar lei com vício de iniciativa, o diploma será inválido.\\
    Entende-se, contudo, que, se um projeto de iniciativa reservada do Presidente da República é apresentado por parlamentar, mas, durante a tramitação da matéria, o Presidente da República envia projeto de lei ``substancialmente idêntico ao que se encontrava em curso no Congresso Nacional”, estará revelada a sua plena e inequívoca vontade de deflagrar o processo legislativo, ``ficando atendida a exigência constitucional da iniciativa” \cite[p.~1062]{mendes_curso_2024}
\end{quote}

\subsection{Veto}

O veto deve obrigatoriamente ser expresso e fundamentado:

\begin{enumerate}[(a)]
    \item na inconstitucionalidade do projeto \textbf{(veto jurídico)}; ou
    \item Na contrariedade ao interesse público \textbf{(veto político)}.
\end{enumerate}

O presidente dispõe de prazo de \textbf{15 dias úteis} para apor o veto, findo o qual é inconstitucional apô-lo. O veto é irretratável e pode ser \textbf{total} ou \textbf{parcial}.

\begin{quote}
    O veto parcial não pode recair apenas sobre palavras ou conjunto de palavras de uma unidade normativa. O veto parcial não pode deixar de incidir sobre o texto integral de artigo, parágrafo, inciso ou alínea. Busca-se prevenir, assim, a desfiguração do teor da norma, que poderia acontecer pela supressão de apenas algum de seus termos \cite[p.~1062]{mendes_curso_2024}.
\end{quote}

O veto não é absoluto; é \textit{relativo}. O Cnngresso Nacional pode \textbf{rejeitar} o veto em sessão conjunta ocorrida em 30 dias da comunicação do veto a ele. Exige-se maioria absoluta dos deputados e maioria absoluta dos senadores para que o Congresso mantenha o projeto que votou. Assim como o veto em si, sua rejeição também pode ser parcial.

Não há veto ou sanção na emenda à Constituição, em decretos legislativos, em resoluções, leis delegadas ou em lei resultante da conversão, sem alterações, de medida provisória.

\section{Promulgação e publicação}

\begin{quote}
    Com a promulgação se atesta a existência da lei, que passou a existir com a sanção ou com a rejeição do veto, e se ordena a sua aplicação. O Presidente da República promulga a lei, mas, no caso da sanção tácita ou da rejeição de veto, se não o fizer em quarenta e oito horas, cabe ao Presidente do Senado a incumbência. A publicação torna de conhecimento geral a existência do novo ato normativo, sendo relevante para fixar o momento da vigência da lei \cite[p.~1064]{mendes_curso_2024}.
\end{quote}

\section{Atividade}

Na qualidade de assessor da Secretaria de Assuntos Legislativos - SAL do Ministério da Justiça, você recebe a notícia de que um deputado federal propus projeto de lei reestruturando a carreira federal dos docentes de ensino superior. Sabendo que a iniciativa dessa matéria é privativa do Presidente da República, mas também que ele concorda com o projeto apresentado pelo deputado, você o orienta a apresentar projeto, com teor idêntico, à Câmara dos Deputados, o que é feito. Durante a tramitação do projeto, um grupo de deputados da base do Presidente apresenta emendas aumentando significativamente os salários dos professores federais. O projeto vai à votação e é aprovado, retornando para você. Sabendo que o Presidente não deseja esse aumento salarial, o que você pode fazer para atendê-lo?

\printbibliography

\end{document}